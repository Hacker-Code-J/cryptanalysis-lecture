\section{2023-Spring-Midterm}

\begin{enumerate}[\bf 1.]
	\item \textbf{[ Substitution Cipher]}
\begin{python}[](style=one-dark){colback=gray!40!black,colframe=blue!50}
Alphabet = 'ABCDEFGHIJKLMNOPQRSTUVWXYZ'
#-- Subst-Encryption
def subst_encrypt(key, msg):
	result = ''
	InSet = Alphabet     # InSet = 'AB...Z'
	OutSet = key 	# OutSet = '...' (Key)
	for ch in msg:
		if ch.upper() in InSet:
			idx = InSet.find(ch.upper())	
			if ch.isupper():
				result += OutSet[idx].upper() 
			else:
				result += OutSet[idx].lower()
		else:
			result += ch
	return result
#-- Subst-Decryption
def subst_decrypt(key, msg):
    result = ''
	InSet = Alphabet
	OutSet = key	
	for ch in msg:
		if ch.upper() in InSet:
			idx = InSet.find(ch.upper())
			if ch.isupper():
				result += OutSet[idx].upper()
			else:
				result += OutSet[idx].lower()
		else:
			result += ch
	return result
\end{python}
\begin{enumerate}[(a)]
	\item \ 
\begin{python}[](style=one-dark){colback=gray!40!black,colframe=blue!50}
# A -> A, B -> B, C -> F, ..., Z -> O
key1 = 'ABFWQICPLZYUDNHMGKEXVTSJRO'
pt1 = 'banana'
ct1 = subst_encrypt(key1, pt1)
print('Eng =', Alphabet)
print('key1 =', key1)
print('pt1 =', pt1)
print('ct1 =', ct1)
\end{python}
\begin{lstlisting}[style=terminal]
Eng = ABCDEFGHIJKLMNOPQRSTUVWXYZ
key1 = ABFWQICPLZYUDNHMGKEXVTSJRO
pt1 = banana
ct1 = banana
\end{lstlisting}
	\item \
\begin{python}[](style=one-dark){colback=gray!40!black,colframe=blue!50}
def valid_key1(key):
	list_key = list(key)
	list_key.sort() # ASCII; A = 65, a = 97
	key_alphabet = ''.join(list_key).upper()
	return key_alphabet == Alphabet

def valid_key2(key):
	for ch in key:
		if not ch.isalpha():
			return False
		if len(set(key)) != 26:
			return False
	return True
\end{python}
\begin{python}[](style=one-dark){colback=gray!40!black,colframe=blue!50}
key_b1 = 'AAADEFGHIJKLMNOPQRSTUVWXYZ'
key_b2 = 'ABCDEFGHIJKLMNOPQRSTUVWxyz'
key_b3 = 'abcDEFGHIJKLMNOPQRSTUVWXYZ'

print(f"b1 | {valid_key1(key_b1)} : {valid_key2(key_b1)}")
print(f"b2 | {valid_key1(key_b2)} : {valid_key2(key_b2)}")
print(f"b3 | {valid_key1(key_b3)} : {valid_key2(key_b3)}")
\end{python}
\begin{lstlisting}[style=terminal]
b1 | False : False
b2 | True  : True
b3 | False : True
\end{lstlisting}
	\newpage
	\item \ 
\begin{python}[](style=one-dark){colback=gray!40!black,colframe=blue!50}
def gen_random_key(seed):
	random.seed(seed)
	alpha_list = list(Alphabet)
	random.shuffle(alpha_list)
	shuffled_key = ''.join(alpha_list)
	return shuffled_key
\end{python}
\begin{python}[](style=one-dark){colback=gray!40!black,colframe=blue!50}
def gen_random_key(seed):
	random.seed(seed)
	alpha_list = list(Alphabet)
	valid_key = False
	while not valid_key:
		random.shuffle(alpha_list)
		shuffled_key = ''.join(alpha_list)
		valid_key = True
		for i in range(len(Alphabet)):
			# if A -> A or ... or Z -> Z
			if Alphabet[i] == shuffled_key[i]:
				valid_key = False
				break # for-break
		return shuffled_key
		
for _ in range(10):
	alpha_list = list(Alphabet)
	for _ in range(255):
		random.shuffle(alpha_list)
	seed = ''.join(alpha_list)
	print(gen_random_key2(seed))
\end{python}
\begin{lstlisting}[style=terminal]
QODUNVELJPFZGTYCXSAKRIBHMW
NRHSXVFWUCYOJZBTLMPEKDAIQG
CYKXDVPEBMWAZRJHOTISFUNQLG
TDMZLNUSQCVJOGIEAWRFKXYHPB
LYDJTIPFNRWQKEOMBVHXGUSZAC
HSOUNZPTIJRGQWAXLVYFDEMBKC
EDBICNMHUYQOZXSJRLAWFKPTGV
KUDCFEHXYATINRPVQMGWZSLBJO
CKSHBAPROUVINXTLWFEJGQMYZD
KARZCIFPDUMJTYGHWSQEOVLBXN
\end{lstlisting}
\end{enumerate}
\newpage
\item \textbf{[ Index of Coincidence]}
\begin{tcolorbox}[title=\bf Index of Coincidence, colbacktitle=cyan, colframe=cyan]
	For a given ciphertext, the \textbf{index of coincidence} $I$ is defined to be the probability that two randomly selected letters in the ciphertext represent the same plaintext symbol.
	\hspace{12pt}For a given ciphertext, let $n_0,n_1,...,n_{25}$ be the respective letter counts of \texttt{A}, \texttt{B}, \texttt{C}, $\dots$, \texttt{Z} in the ciphertext, and set $n=n_0+n_1+\cdots+n_{25}$. Then, the index of coincidence can be computed as \[
	I=\frac{\binom{n_1}{2}+\binom{n_2}{2}+\cdots\binom{n_{25}}{2}}{\binom{n_1+\cdots+n_{25}}{2}}=\frac{\sum_{i=0}^{25}\binom{n_i}{2}}{\binom{n}{2}}=\frac{1}{n(n-1)}\sum_{i=0}^{25}[n_i(n_i-1)].
	\]
	\hspace{12pt} To see why the index of coincidence gives us useful information, first note that the empirical probability of randomly selecting two identical letters from a large English plaintext is
	\[
	\sum_{i=0}^{25}p_i^2\approx0.065.
	\]
\end{tcolorbox}
	\begin{enumerate}[(a)]
		\item Attack with Index of Coincidence
		\begin{table}[h!]
		\begin{tabular}{rc}
		Caesar Cipher & \texttt{IC(PT)=IC(CT)} \\
		Substitution Cipher & \texttt{IC(PT)=IC(CT)} \\
		Vigen\'{e}re Cipher & \texttt{IC(CT)$\approx$IC(rand)}, \textth{IC(CT)<IC(PT)}
		\end{tabular}
		\end{table}
		\item 알파벳 26 글자 중 `A'의 빈도가 30\%이고, `B'-`K'까지 10글자가 같은
		비율로 사용되며, 나머지 알파벳은 쓰지 않는 언어가 있다고 가정하자. 이 언어로
		된 문서의 `Index of Coincidence'는 얼마인가?
		\begin{proof}[\textcolor{magenta}{\bf Sol}]
			\ \begin{itemize}
				\item Document with $N$ characters.
				\item Frequency of `B' - `K' : 70\% (for each 7\%)
			\end{itemize} Then \[
			IC=\frac{1}{N(N-1)}\left[0.3N(0.3N-1)+0.07N(0.07-1)\times 10\right]
			\] and so \[
			\lim\limits_{N\to\infty}IC=\frac{(0.3)^2\cdot N^2+\cdots + (0.07)^2\cdot10\cdot N^2+\cdots\times}{N^2+\cdots}=0.09+0.049=0.139.
			\]
		\end{proof}
		\newpage
		\item 다음은 영문을 Vigen\'{e}re 암호로 암호화한 문서에 대하여 키 길이 key len
		을 1부터 증가시켜가며 암호문을 키 길이 간격으로 추출한 sub msg에 대하여
		`IC(Index of Coincidence)'를 계산한 결과이다. 아래 결과로부터 사용된 암호키
		는 몇 글자로 추정되는가?
\begin{lstlisting}[style=terminal]
key_len =  1 :IC(sub_msg) = 0.0435
key_len =  2 :IC(sub_msg) = 0.0493
key_len =  3 :IC(sub_msg) = 0.0428
key_len =  4 :IC(sub_msg) = 0.0598
key_len =  5 :IC(sub_msg) = 0.0424
key_len =  6 :IC(sub_msg) = 0.0477
key_len =  7 :IC(sub_msg) = 0.0444
key_len =  8 :IC(sub_msg) = 0.0597
key_len =  9 :IC(sub_msg) = 0.0418
key_len = 10 :IC(sub_msg) = 0.0492
key_len = 11 :IC(sub_msg) = 0.0445
key_len = 12 :IC(sub_msg) = 0.0578
key_len = 13 :IC(sub_msg) = 0.0469
key_len = 14 :IC(sub_msg) = 0.0505
key_len = 15 :IC(sub_msg) = 0.0416
key_len = 16 :IC(sub_msg) = 0.0638
key_len = 17 :IC(sub_msg) = 0.0469
\end{lstlisting}
	\begin{proof}[\textcolor{magenta}{\bf Sol}]
		Length of Key = 4 (Size of Interval)
	\end{proof}
	\end{enumerate}
	\newpage
	\item \
\begin{python}[](style=one-dark){colback=gray!40!black,colframe=blue!50}
#=========================================
# TC1 - Toy Cipher (encryption/decryption)
# - n, k: 32-bit
# - Identical Round Function
# - Key Schedule (X)
#=========================================

NUM_ROUND = 10

#--------------------------------------
#  Encryption
#--------------------------------------

#--- S-Box (AES)
Sbox = [ ... ]
#--- Inverse S-Box (AES)
ISbox = [ ... ]

#-- AR: Add Roundkey
def AR(in_state, rkey):
	out_state = [0] * len(in_state)
	for i in range(len(in_state)):
		out_state[i] = in_state[i] ^ rkey[i] 
	return out_state

#-- SB: Sbox layer
def SB(in_state):
	out_state = [0] * len(in_state)
	for i in range(len(in_state)):
		out_state[i] = Sbox[in_state[i]]
	return out_state

#-- LM: Linear Map
# |Y0|     |0 1 1 1||x0|   |x1^x2^x3|
# |  |     |       ||  |   |        |
# |Y1|     |1 0 1 1||x1|   |x0^x2^x3|
# |  |  =  |       ||  | = |        |
# |Y2|     |1 1 0 1||x2|   |x0^x1^x3|
# |  |     |       ||  |   |        |
# |Y3|     |1 1 1 0||x3|   |x0^x1^x2|
def LM(in_state):
	out_state = [0] * len(in_state)
	All_Xor = in_state[0] ^ in_state[1] ^ in_state[2] ^ in_state[3]
	for i in range(len(in_state)):
		out_state[i] = All_Xor ^ in_state[i]
	return out_state

#-- Enc_Round
def Enc_Round(in_state, rkey):
	out_state = [0] * len(in_state)
	out_state = AR(in_state, rkey)
	in_state = SB(out_state)
	out_state = LM(in_state)
	return out_state

#- TC1 Encryption
def TC1_Enc(PT, key):
	NROUND = NUM_ROUND # Number of Round = 10
	CT = PT #CT = [0] * len(PT)
	for i in range(NROUND):
		CT = Enc_Round(CT, key)
	return CT

#--------------------------------------
#  Decryption
#--------------------------------------

#-- SB: Sbox layer
def ISB(in_state):
out_state = [0, 0, 0, 0]
	for i in range(len(in_state)):
		out_state[i] = ISbox[in_state[i]]
	return out_state

#-- Decrypt Round
def Dec_Round(in_state, rkey):
	out_state1 = [0, 0, 0, 0]
	out_state2 = [0, 0, 0, 0]
	out_state3 = [0, 0, 0, 0]
	out_state1 = LM(in_state)
	out_state2 = ISB(out_state1)
	out_state3 = AR(out_state2, rkey)
	return out_state3

#- TC1 Decryption
def TC1_Dec(input_state, key):
	state = input_state
	numRound = NUM_ROUND
	for i in range(0, numRound):
		state = Dec_Round(state, key)
	return state
\end{python}\\
\newpage
\begin{enumerate}
	\item 암호키 32비트가 \texttt{[0x00, 0x02, 0x01, 0x03]}으로 설정되었다고 하자.
	1라운드 출력의 처음 3바이트가 동일한 입력의 예를 하나만 만들면? \[
	\left[p_0, p_1, p_2, p_3\right]\implies\left[a,a,a,b\right]
	\]
	\begin{proof}[\textcolor{magenta}{\bf Sol}]
		Consider $[a,a,a,b]=[0,0,0,1]$. Then $[p_0, p_1, p_2, p_3]=[\texttt{0x09},\texttt{0x0b},\texttt{0x08},\texttt{0x51}]$
		\begin{center}
		\begin{tikzcd}
			p_0=\texttt{0x09}                      & p_1=\texttt{0x0b}                      &                                                                                      & p_2=\texttt{0x08}                      & p_3=\texttt{0x51}                      \\
			\oplus\texttt{0x00} \arrow[u, no head] & \oplus\texttt{0x02} \arrow[u, no head] &                                                                                      & \oplus\texttt{0x01} \arrow[u, no head] & \oplus\texttt{0x03} \arrow[u, no head] \\
			\texttt{0x09} \arrow[u, no head]       & \texttt{0x09} \arrow[u, no head]                &                                                                                      & \texttt{0x09} \arrow[u, no head]       & \texttt{0x52} \arrow[u, no head]       \\
			S \arrow[u, no head]                   & S \arrow[u, no head]                   &                                                                                      & S \arrow[u, no head]                   & S \arrow[u, no head]                   \\
			1 \arrow[u, no head]                   & 1 \arrow[u, no head]                   &                                                                                      & 1 \arrow[u, no head]                   & 0 \arrow[u, no head]                   \\
			&                                        & LM \arrow[llu, no head] \arrow[lu, no head] \arrow[ru, no head] \arrow[rru, no head] &                                        &                                        \\
			a=0 \arrow[rru, no head]                 & a=0 \arrow[ru, no head]                  &                                                                                      & a=0 \arrow[lu, no head]                  & b=1 \arrow[llu, no head]                
		\end{tikzcd}
	\end{center}
	\end{proof}
	\item Note that \[
	\begin{bmatrix}
		y_0\\ y_1\\ y_2\\ y_3
	\end{bmatrix}=\begin{bmatrix}
	0 & 1 & 1 & 1 \\ 1 & 0 & 1 & 1 \\ 1 & 1 & 0 & 1 \\ 1 & 1 & 1 & 0
\end{bmatrix}\begin{bmatrix}
	x_0 \\ x_1 \\ x_2 \\ x_3
\end{bmatrix}=M\begin{bmatrix}
	x_0 \\ x_1 \\ x_2 \\ x_3
\end{bmatrix}.
	\] Let $M=I_{4}=\begin{bmatrix}
		1 & 0 & 0 & 0 \\ 0 & 1 & 0 & 0 \\ 0 & 0 & 1 & 0 \\ 0 & 0 & 0 & 1
	\end{bmatrix}$. Then $c_0=f_0(p_0, k_0)$ and so we can find $k_0$ with $2^8={256}$ complexity.
\end{enumerate} % Problem3
\newpage
\item \textbf{[ Double Encryption ]} \begin{itemize}
	\item Block Cipher \texttt{E1}: in/out 64-bit, key 64-bit
	\item Block Cipher \texttt{E2}: in/out 64-bit, key 52-bit
	\item Formula: $CT=E2(E1(PT,K1), K2)$
\end{itemize}
	\begin{center}
	\begin{tikzcd}
		PT \arrow[d, "64"'] &                     \\
		E1 \arrow[d, "64"'] & K_1 \arrow[l, "64"] \\
		E2 \arrow[d, "64"'] & K_2 \arrow[l, "56"] \\
		CT                  &                    
	\end{tikzcd}
	\end{center}
	\begin{enumerate}[(a)]
		\item (평문, 암호문) 쌍을 여러 개 수집한 후 공격을 시작한다고 가정하자.
		수집한 평문을 E1으로 암호화하여, 중간값을 저장하고, 수집한 암호문을 E2로
		복호화하여 비교하는 방식으로 공격한다면, 이 때, 사용되는 메모리와 계산량은
		각각 얼마인가?
		\begin{proof}[\textcolor{magenta}{\bf Sol}]
		Memory = $2^{k_1}=2^{64}$, Time = $2^{k_2}=2^{56}$.
		\end{proof}
		\item (a)와 동일한 조건에서 블록암호 E1과 E2의 역할을 반대로 하여 공격한다
		면, 즉, $PT = E1^{-1} (E2^{-1} (CT, K2), K1)$를 이용하여 같은 공격을 수행한다면,
		이 때, 사용되는 메모리와 계산량은 각각 얼마인가?
		\begin{proof}[\textcolor{magenta}{\bf Sol}]
		Memory = $2^{k_2}=2^{56}$, Time = $2^{k_1}=2^{64}$.
		\end{proof}
		\item 높은 확률로 하나의 키 후보만 남도록 하려면 몇개의 (평문, 암호문) 쌍을
		수집해야 하는가?
		\begin{proof}[\textcolor{magenta}{\bf Sol}]
			\ \begin{itemize}
				\item 64-bit 블록암호에서 우연히 중간값이 일치할 확률 = $\frac{1}{2^{64}}$
				\item $2^{k_1}$ (테이블 크기, 저장 되어 있는 중간값의 개수)
				\item $2^{k_2}$ (비교 횟수)
			\end{itemize} Thus, \[
			\frac{1}{2^{64}}\times 2^{k_1}\times 2^{k_2}=2^{k_1+k_2-64}=2^{64+56-64}=2^{56}.
			\] 추가로 (PT,CT) 한 쌍을 확인하면 $2^{56}\times\frac{1}{2^{64}}=\frac{1}{8}$.
			MIMT + 한쌍 or 처음부터해서 두 쌍.
		\end{proof}
		\item 선택평문공격(chosen plaintext attack)이 가능하다면, 고정된 평문에 대
		한 사전계산(pre-computation)을 이용하여 공격할 수 있다. 이 때 최적의 공격
		방법을 구성하고 사전계산으로 얻어지는 장점을 설명하라.
		\begin{proof}[\textcolor{magenta}{\bf Sol}]
			\ \begin{enumerate}
				\item 고정 평문에 대하여 $2^{k_1}$ 테이블을 사전계산한다.
				\item 암호문 수집 후 $2^{k_2}$ 정도의 복호화하면서 찾을 수 있다.
			\end{enumerate}
		\end{proof}
	\end{enumerate} %Problem 4
\item \textbf{[ TMTO Attack ]}
\begin{enumerate}[(1)]
	\item $m$: number of random staring points
	\item $t$: length of chain
	\item $\ell$ : number of Hellman Table \[
	X_{i,j+1}=f(X_{i,j})=E(PT,X_{i,j}),\quad(i=1,2,\dots,m, j = 0,1,\dots,t).
	\]
\end{enumerate} Let $m=2^{24},t=2^{20}$ and $\ell=2^{20}$.
\begin{center}\begin{tikzcd}
		{SP_1=X_{1,0}} \arrow[r] & {X_{1,1}} \arrow[r] & {X_{1,2}} \arrow[r] & \cdots \arrow[r] & {X_{1,t-1}} \arrow[r] & {X_{1,t}=EP_1} \\
		{SP_2=X_{2,0}} \arrow[r] & {X_{2,1}} \arrow[r] & {X_{2,2}} \arrow[r] & \cdots \arrow[r] & {X_{2,t-1}} \arrow[r] & {X_{2,t}=EP_2} \\
		{SP_i=X_{i,0}} \arrow[r] & {X_{i,1}} \arrow[r] & {X_{i,2}} \arrow[r] & \cdots \arrow[r] & {X_{i,t-1}} \arrow[r] & {X_{i,t}=EP_i} \\
		{SP_m=X_{m,0}} \arrow[r] & {X_{m,1}} \arrow[r] & {X_{m,2}} \arrow[r] & \cdots \arrow[r] & {X_{m,t-1}} \arrow[r] & {X_{m,t}=EP_m}
	\end{tikzcd}
\end{center}
\begin{enumerate}[(a)]
	\item 암호문 CT 에 f 를 a번 적용하여 EPi와 일치함을 확인했다면, 암호키를
	어떻게 결정하는가? 이 과정에서 필요한 계산량은?
	\[EP_i(\underbrace{f\circ f\circ\dots\circ f}_{a})(CT)\]
	\begin{proof}[\textcolor{magenta}{\bf Sol}]
	$SP_i=X_{i,0}$를 $(t-a-1)$번 암호화 하면 $X_{t-a-1}(=key)$를 얻는다.
	\end{proof}
	\item 테이블 준비에 필요한 사전 계산량(pre-computation)은 얼마인가? 전수
	조사보다 많은 사전 계산량을 사용해도 의미있는 공격이 가능함을 설명하라.
	\begin{proof}[\textcolor{magenta}{\bf Sol}]
	$m\times t\times l=2^{24}\times 2^{20}\times 2^{20}=2^{64}$. 실제 암호문 획득 이전에 미리 테이블을 만들어두면 실제 암호문을 획득 했을 때 해당되는 키를 빠르게 얻을 수 있다.
\end{proof}
	\item 사전계산 후 공격에 필요한 메모리와 계산량은 각각 얼마인가?
	\begin{proof}[\textcolor{magenta}{\bf Sol}]
	사전 계산량 $mtl$를 제외한다면 $M=m\times l$ and $T=t\times l$.
	\end{proof}
	\item 높은 확률로 하나의 키 후보만 남기도록 하려면, 추가적으로 필요한 (평문,
	암호문)의 쌍은 몇 개인가?
	\begin{proof}[\textcolor{magenta}{\bf Sol}]
		다른 한 쌍 ($P',C'$)를 추가적으로 더 확인해보면 높은 확률로 하나의 키 후보를 남기게 된다.
	\end{proof}
\end{enumerate} % Problem 5
\end{enumerate}

%\newpage
%
%\begin{proof}[\textcolor{magenta}{\bf Sol}]
%\end{proof}
%\begin{python}[](style=one-dark){colback=gray!40!black,colframe=blue!50}
%\end{python}
%\begin{python}[](style=one-dark){colback=gray!40!black,colframe=blue!50}
%\end{python}
%\begin{python}[](style=one-dark){colback=gray!40!black,colframe=blue!50}
%\end{python}
%\begin{python}[](style=one-dark){colback=gray!40!black,colframe=blue!50}
%\end{python}
%\begin{python}[](style=one-dark){colback=gray!40!black,colframe=blue!50}
%\end{python}
%\begin{python}[](style=one-dark){colback=gray!40!black,colframe=blue!50}
%\end{python}
%\begin{python}[](style=one-dark){colback=gray!40!black,colframe=blue!50}
%\end{python}
%\begin{python}[](style=one-dark){colback=gray!40!black,colframe=blue!50}
%\end{python}
%\begin{python}[](style=one-dark){colback=gray!40!black,colframe=blue!50}
%\end{python}
%\begin{python}[](style=one-dark){colback=gray!40!black,colframe=blue!50}
%\end{python}
%\begin{python}[](style=one-dark){colback=gray!40!black,colframe=blue!50}
%\end{python}